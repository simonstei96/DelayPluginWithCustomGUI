\chapter{Zusammenfassung und Ausblick}

Mit dem JUCE Framework ist es möglich ein Delay Plugin zu erstellen. Dieses funktioniert ohne Knackgeräusche wenn während der Nutzung der Delaytimewert nicht verändert wird. Folglich empfiehlt sich also nicht den Wert in einer DAW zu automatisieren. 

Die Erstellung eigener UI-Komponenten ist auch relativ einfach möglich. Diese können entweder aus weiteren vorgefertigten Elementen bestehen oder eigene zeichnen wie in diesem Fall den Trackball und das Begrenzungsfeld.

Diese Variante könnte in der Zukunft weiter optimiert (Interpolation) oder um weitere Funktionen erweitert werden. Eine mögliche Erweiterung ist ein unterschiedliches Delay pro Kanal, so dass einer eine Verzögerung von 500ms und der andere von 550ms hat.